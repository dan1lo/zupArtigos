\documentclass{TheMartianReport}
\usepackage{lipsum}

%%%%%%%%%%%%%%%%%%%%%% COVER PAGE VALUES %%%%%%%%%%%%%%%%%%%%%%
\title{Case Study Protocol} % Change document title
\author{Danilo Monteiro Ribeiro, Victor Hugo Santiago and Alberto Souza } % Change author name
\project{A Protocol About Self Efficacy In Software Engineering } % Change (or omit) project name
%\date{} % Change displayed date value from default
\imagefolder{./examples} % Change folder to look for images in
%\logo{} % Add logo image to cover page
\revision{Revision 1.0} % Change revision number

%%%%%%%%%%%%%%%%%%%%%%% GENERAL SETTINGS %%%%%%%%%%%%%%%%%%%%%%%
%\bibliographystyle{abbrvnat} % Change bibliography style
%\footertext{}{} % Change values in footer text
%\nofootertext % Hide footer text
\imageref{Header image is "Space Shuttle Columbia launching" from NASA Commons} % Reference where header images came from (appear on last page as footnote)
\drafting % Comment out to hide \temp and \crit sections
%\sectionnumbers% Sections are major numbers, subsections minor, etc.
%\subsectionnumbers % Sections not numbered, only begin at subsections and below

%%%%%%%%%%%%%%%%%%%%%%%%%%%%%%%%%%%%%%%%%%%%%%%%%%%%%%%%%%%%%%%%%%%%%%%%%%
\begin{document}

%\printhelp % Use this to see/print help pages

\subsection{Executive Summary}
This is structure of a Rapid Review that will be used at Zup Innovation. 

\section{Introduction}{sectionimage.jpg}
This is a document about how we can perform a Rapid Review at Zup Innovation. This is a guideline with some information with main process are used to create Rapid Review.


Sometimes, we will not use references at our report because our main objective is not teach you with all scientific rigor about Rapid Review, instead of we are trying to explain how to conduct a rapid review.Be aware about this!

 For more academic information, please look:
 
\begin{nicelist}{List of main references }
\item \cite{cartaxo2018model}
\item \cite{cartaxo2018role}
\item \cite{cartaxo2020rapid}
\end{nicelist}


\subsection{Why use ?}





\section{ The process}

First of all, the Rapid Review has same process as the Systematic Review but at some steps the process is more light. For example, when you conduct a Systematic Review, you look at more than one search engine. But when you 



\temp{These are some notes for yourself, designed for use when dummying out sections, only visible when you put \textbf{\textbackslash drafting} in the preamble. You can comment it out or remove it to not see this.}
\lipsum[50]
\crit{These are some other notes for yourself, designed for use when critiquing sections, only visible when you put \textbf{\textbackslash drafting} in the preamble. You can comment it out or remove it to not see this.}
\lipsum[100]
\tipbox*{Note}{This is a wrapped tipbox. This is text that goes in the note, let me see if it wraps and still looks nice?}
\lipsum

\subsection{Second Subsection Name}
\lipsum[100]
\tipbox{Alert!}{This is a non-wrapped tipbox (no paragraph after it). This is text that goes in the note, let me see if it wraps and still looks nice?}

\subsubsection{Subsubsection Name}
\lipsum[80]
\begin{niceenum}{Enum Title}
	\item This is an item
	\item This is a second item	
	\item \link{secretlab.com.au}{This item is a link}
\end{niceenum}
\lipsum[40]

\subsubsubsection{Subsubsubsection Name}
\lipsum[100]


\subsubsubsubsection{Subsubsubsubsection Name}
\lipsum[40]

\subsubsubsubsubsection{Subsubsubsubsection Name}
\lipsum[40] Please see Figure \ref{img:logo.png}.

\image{0.4}{latex.png}{Caption Text}

\subsubsubsubsubsubsection{Subsubsubsubsubsection Name}
\lipsum[100]

\subsubsubsubsubsubsubsection{Subsubsubsubsubsubsection Name}
\lipsum[100] As per \cite{kopka2003guide}, \cite{mittelbach2004latex}, \cite{hahn1993latex}, \cite{long2015concept}, and \cite{griffiths1997learning}.

\bibliography{citations.bib}

\end{document}
%%%%%%%%%%%%%%%%%%%%%%%%%%%%%%%%%%%%%%%%%%%%%%%%%%%%%%%%%%%%%%%%%%%%%%%%%%
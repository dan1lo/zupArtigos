\documentclass{TheMartianReport}
\usepackage{lipsum}

%%%%%%%%%%%%%%%%%%%%%% COVER PAGE VALUES %%%%%%%%%%%%%%%%%%%%%%
\title{Understanding self-efficacy in Software Engineering: A Rapid Review } % Change document title
\author{Danilo Monteiro Ribeiro, Victor Hugo Santiago and Alberto Souza } % Change author name
\project{A Rapid Review Protocol} % Change (or omit) project name
%\date{} % Change displayed date value from default
\imagefolder{./examples} % Change folder to look for images in
%\logo{} % Add logo image to cover page
\revision{Revision 1.0} % Change revision number

%%%%%%%%%%%%%%%%%%%%%%% GENERAL SETTINGS %%%%%%%%%%%%%%%%%%%%%%%
%\bibliographystyle{abbrvnat} % Change bibliography style
%\footertext{}{} % Change values in footer text
%\nofootertext % Hide footer text
\imageref{Header image is "Space Shuttle Columbia launching" from NASA Commons} % Reference where header images came from (appear on last page as footnote)
\drafting % Comment out to hide \temp and \crit sections
%\sectionnumbers% Sections are major numbers, subsections minor, etc.
%\subsectionnumbers % Sections not numbered, only begin at subsections and below

%%%%%%%%%%%%%%%%%%%%%%%%%%%%%%%%%%%%%%%%%%%%%%%%%%%%%%%%%%%%%%%%%%%%%%%%%%
\begin{document}

%\printhelp % Use this to see/print help pages

\subsection{Executive summary}
This is a summary protocol of research about Self efficacy in Software Engineering. The main objective of this protocol is documented the steps that are used to find studies and answers that can help Zup, and more specifically the Zup Academy to help zuppers to exponential growth.  

\section{INTRODUCTION}{sectionimage.jpg}

vdZup Innovation is always looking how to improve the team performance and promote exponential growth for their collaborators. Based on this, zup is seeking to better understand the impact of self-efficacy on its software development teams and members.

 First, because Self-efficacy is constant associate to learn and performance (\cite{bandura1999self}). Second, Self-efficacy is also associate to other factors like more satisfaction ( \cite{domenech2017self}, more motivation to quit and less burnout ( \cite{federici2012principal}).

Those factors are interest for Zup.
As said earlier, Exponential growth is one of pillars of Zup and understand how we can improve by learning and coaching the zuppers' performance is important. Otherwise, if zuppers improve their performance, others good consequents can appear to organization, like more quality, more time to market and others (\cite{bandura2000cultivate}).

Despite this, the Zup and Software Engineering Research community do not know (briefly)  the antecedents and consequents of self-efficacy in software Engineering. More than that, How zup can measure self-efficacy in software engineering. To understand this, we conduct this rapid review. 

Therefore, this document contains a rapid review protocol about self-efficacy in Software Engineering. Our main objective is understand \textbf{what we know about self-efficacy in Software Engineering?} More specifically, we are looking for understand:
 
\begin{nicelist}{Research Questions}
\item What are the results about self-efficacy in Software Engineering?
\item What are the antecedents and consequents of self-efficacy in Software Engineering?  
\item What are the Self-efficacy investigated? 
\item What are the scales and questionnaires used to understand self-efficacy in Software Engineering? 
\item How (context/method) the studies investigates self-efficacy in Software Engineering ?
\end{nicelist}






\section{Research steps}

\subsection{Search Strategy}

\cite{lenberg2015behavioral} conducted a systematic review about several psychological factors that can impact software development. Among them was self-efficacy. They found 14 studies that investigated self-efficacy in software engineering. 

We use the same search string used by \cite{lenberg2015behavioral} ((”self efficacy” OR ”self-efficacy”) and Software Engineering) at IEEE. We found 66 studies.

\subsection{Inclusion and exclusion criteria}
We select the studies based at follow criteria:


\textbf{Inclusion Criteria}

\textit{Publication Year:} We limited the search to include publications to August 2020.

\textit{Publication Type:} We choose to include papers published both in journals, in
conference proceedings and in workshop proceedings.

\textit{Content:} Self-efficacy had to have been studied in relation Software Engineering activities or to software engineers.  


\textbf{Exclusion Criteria}

\textit{Language:} We limited this study to only include papers written in English.

\textit{Publication Type:} We excluded papers where we could not locate a full paper
version or the paper is not available to download for us ( although very few papers were affected by this exclusion criterion).


We will look for papers that based at title and abstract. If some study had potential, we will included for the second step.

The second step is  read of introduction and conclusion of those studies. If study is approved, it will be part of our "studies set".  


\subsection{Extraction step}
To extraction information from studies we will use the google spreadsheet. We extracted: 

(1) Paper name, (2) source, (3) Authors, (4)Definitions, (5) Method, (6) Sample, (7) Sample Context (8) Scales/Questionnaires, (9) Antecedents, (10) Consequents, (11) Research Questions, (12) Results, .


\subsection{Quality Step}

We prefer do not conduct the quality step as recommended by \cite{cartaxo2020rapid}. This occurs because our results is heterogeneous as to the methods, and constructs evaluated at studies. 

\section{Expected results}

With this rapid review we expected some results: Understand the main results, consequents,antecedents,scales, questionnaires, and research design used to conducted studies about self-efficacy in software Engineering.

With those results, Zup, 
specifically Zup Academy, can understand why/how we need to measure and promote self-efficacy,
 to conduct other internal studies to understant the impact of self-efficacy at zuppers. 






\bibliography{citations.bib}

\end{document}
%%%%%%%%%%%%%%%%%%%%%%%%%%%%%%%%%%%%%%%%%%%%%%%%%%%%%%%%%%%%%%%%%%%%%%%%%%